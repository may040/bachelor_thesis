\chapter{Verwandte Arbeiten}
Die Privatisierung von Daten durch \gls{dp} wird nicht nur in der Medizin seit einigen Jahren verwendet, sondern auch in der Industrie wie z.B. von Apple für die Auswertung von Nutzern bei der Versendung von Emojis \parencite{Apple}. Diese Arbeit beschränkt sich auf das Gesundheitswesen und die Evaluationsmöglichkeiten von \gls{dp}.


Im Jahr 2014 forschten gemeinsam nach einem Ansatz Wissenschaftler Raquel Hill, Michael Hansen, Erick Janssen und weitere aus verschiedenen Universitäten \parencite{Hill2014} nach einem Ansatz für die Nutzbarkeit von \gls{dp} für wissenschaftlichen Daten. Der Datensatz über ungeschützten Geschlechtsverkehr und ungeplante Schwangerschaften von der Kinsey Institut bereitgestellt. Diese Daten wurden durch \gls{dp} mit verschiedenen Algorithmen wie zellen-basiert, kd-Baum-Partitionierung, Privolet-Algorithmus (ad-hoc) und Walvet (basic) generiert. Die Metriken der Evaluation dieser verrauschten Daten in der Nutzbarkeit unterteilten sich in zwei Kategorien: multivariate logistische Regression und die Wichtigkeit der Merkmale der Daten (z.B. Dimension). 
Zusätzlich werden vier verschiedene Methoden k-d Baum, Basis und adhoc (Privelet Algorithmus) für \gls{dp} angewandt, um das Rauschen für die Zellen und die Histogramme zu erzeugen.
Für die Evaluation wurde der Datensatz in zwei Teilmengen unterteilt. Alle drei Daten, der gesamte Datensatz (MART\_FINAL) und die zwei Teilmengen (MART\_RS1, MART\_RS2), wurden durch \gls{dp} mit den verschiedenen Methoden ausgeführt und das Chancenverhältnis (in Englisch odds ratio) aus der logistischen Regression verglichen. Schlussendlich wurde festgestellt, dass \gls{dp} hat mit dem großen Unterschied zwischen der Anzahl von Zeilen und der gedeckten Zellen durch die Abfrage in der Dimension (zweite Kategorie) sehr große Schwierigkeiten hat. Dabei der gesamte Datensatz bei der Evaluation stets schlechter ab. 

In dieser Bachelorarbeit wird zur Kompatibilität der Frameworks die Nutzung eines großen Datensatzes anstatt mehrere. Es findet keine Unterteilung des Datensatzes statt.

Ein interessanter Einsatz von \gls{dp} ist die Privatisierung von Texten \parencite{TEM2021}. Hier wird der Algorithmus \enquote{\gls{tem}} entwickelt, welcher eine höhere Nutzbarkeit als der state-of-art (Stand 2021) haben. Der Algorithmus ermöglicht für eine beliebige Distanzmetrik das Privatisieren von Wörtern. Die Besonderheit liegt in der spezifischen Ausführung des \enquote{Exponential mechanism}, welcher für den Entwurf eines \gls{dp} Algorithmus genutzt wird. Der Schritt zur Privatisierung wird zu einem Auswahlproblem verlegt. Das Eingabewort wird durch eines mit einer höheren Ähnlichkeit zum Eingabewort aus der Worteinbettung ersetzt. Das Rauschen kann dadurch abhängig von der Dichte der Worteinbettung hinzugefügt werden. In ihren Experimenten mit einem vergleichbaren Algorithmus erreicht \gls{tem} eine deutliche höhere Genauigkeit und Privatisierung beim selben Privatisierungsparameter ($\epsilon$).


Dieses Paper befasst sich in Gegensatz zu dieser Bachelorarbeit mit dem Themengebiet Machine Learning in Gegensatz zu dieser Bachelorarbeit, jedoch wird folgendes für die Evaluation der Frameworks deutlich: Die Genauigkeit und der Privatisierungsgrad ist durch eine Metrik messbar, eine Verbesserung der Resultate ist durch den Mechanismus für \gls{dp} entscheidend und die Weise der Hinzufügung des Rauschens zu den Daten spielt eine wichtige Rolle. In den drei evaluierten Frameworks werden die bekannten Mechanismen wie Laplace und Gauss selbst implementiert und eingesetzt bzw. von einer Third Party verwendet. Des Weiteren ist die Zuteilung des Rauschens bei ihnen unbekannt und kann eine erhebliche Auswirkung auf die Genauigkeit haben.


Des Weiteren gelang es den Wissenschaftlern der Pennsylvania State University in den USA wie ein \gls{dp} Framework, welches von Herrn Wasserman und Zhou entwickelt worden sind \parencite{DPFramework}, abgeändert werden muss, um für klinische Daten einsetzbar zu sein \parencite{Duy2009}. In solchen Anwendungsgebieten sind für Untersuchungen Hypothesen, welche üblicherweise statistisch getestet und modelliert werden, durch dieses Framework ermöglicht worden. Die Stichprobe ist entscheidend für die Effizienz und den Grad an Privatisierung der Daten. Beim Algorithmus von \gls{dp} wurde die Zerlegung in Gruppen für eine bessere Schätzfunktion durchgeführt und beim Laplace Mechanismus ein Parameter unterlassen, so dass bei einem festen Stichprobenumfang $N$ nur das $\epsilon$ anpasst werden muss, um das Verhältnis zwischen der Privatsphäre und den asymptotischen Eigenschaften des zurückgegebenen Schätzers zu bestimmen. Durch Experimente wurde gezeigt, dass dadurch der Mean Squared Error von \gls{dp} deutlich niedriger als die eines Maximum-Likelihood-Schätzers für dieselben Daten ist. Dies ermöglicht sensible medizinische Daten unter Einhaltung der Privatsphäre effizienter auszuwerten.


In dieser Bachelorarbeit wird ebenfalls die Einsetzbarkeit von \gls{dp} als Framework für medizinische Daten untersucht und evaluiert. Der Datensatz wird dagegen nicht unterteilt, sondern in der gleichen Größe für alle Frameworks bereitgestellt. Des Weiteren werden die Frameworks unverändert entsprechend der Definition von \gls{dp} eingesetzt. Diese Arbeit wird dem realistischen Einsatz von \gls{dp} für das Gesundheitswesen aus dem Paper nachgehen.

Unterstützt von der EURECOM sowie privaten Firmen entwickeln Eleonora Ciceri , Marco Mosconi, Melek Önen und Orhan Ermis die Plattform PAPAYA, um Datenschutzbedenken auszuräumen, wenn Datenanalyseaufgaben von nicht vertrauenswürdigen Datenverarbeitern durchgeführt werden. Dieses Projekt soll unter Einhaltung der DSGVO und des Schutzes der Kunden durchgeführt werden, während wertvolle und aussagekräftige Informationen aus den analysierten Daten extrahiert werden. Das PAPAYA-Projekt fokussiert sich auf drei Hauptdatenanalysetechniken: neuronale Netze (Training und Klassifizierung), Clusterbildung und grundlegende Statistiken (Zählung). PAPAYA widmet sich zwei Anwendungsfällen im Bereich der digitalen Gesundheit. Im ersten Bereich für Erkennung von Arrhythmie und Stress. Im zweiten Fall wird \gls{dp} für das Zusammenführen der personenbezogenen medizinischen Daten Angestellter für das neuronale Netzwerk eingesetzt. Einzelne Angestellte sind somit vor einer Veröffentlichung ihrer Daten bewahrt und das Netzwerk kann trotz dessen Berechnungen durchführen.


Dieser zweite Anwendungsfall entspricht dem dieser Bachelorarbeit in der Konstruktion. Den Patienten wird ermöglicht durch ihre \gls{ePa} ihre Daten der Forschung bereitzustellen, welche durch eine Datenbank zusammengeführt und anschließend durch \gls{dp} zur privatisierten Auswertung vorhanden sind. Die Frameworks basieren weder auf Machine Learning noch beinhalten solche Funktionen im Themenbereich der Klassifizierung. Sie bieten generelle statistische Funktionen für \gls{dp} Brechungen an. Des Weiteren gilt ebenfalls die Einhaltung der DSGVO für die Frameworks für den medizinischen Anwendungsfall.


Eine große und ausführliche Übersicht über die bisherigen Ergebnisse der Forschung bezüglich \gls{dp} bildet das Paper \parencite{OverviewDP} der CHEO Research Instituts. In diesem wird über die Begrenzungen des Verfahrens sowie seine Mechanismen (Laplace, Gauß usw.) gesprochen, die durch die Forschung zum Vorschein kam. Unter anderem sind Schwierigkeiten mit den Anfragen an Datenbanken aufgekommen, da die Genauigkeit mit vielen verschiedenen Anfragen abnimmt. Sonst kann ein Angreifer durch die verrauschten Ausgaben eine Fraktion einer Datenbank rekonstruieren. Des Weiteren folgt, dass das Rauschen aus dem Mechanismus linear mit der Anzahl der Anfragen wächst. Dies impliziert, dass bei sublinearem Rauschen des Mechanismus nicht in der gleichen Menge Anfragen beantworten kann. Als Resultat wird festgestellt, der Laplace Mechanismus garantiert die Beantwortung zufälliger Anfragen, ist effizient, erreicht $\epsilon$-\gls{dp} und beantwortet Anfragen anpassungsfähig. Jedoch erfüllt er das Kriterium, eine große Anzahl an Anfragen mit einer nicht einfachen Genauigkeit zu beantworten, nicht. Für die Anwendung von \gls{dp} im Gesundheitswesen, sind gewisse Umstände sehr zu beachten. Bei Anfragen für die Anzahl der Patienten mit einer gewissen Eigenschaft, ist nur durch ein abgewandelter Exponentieller Mechanismus erfolgreich. Eine vorgegebene Menge an Budget für den Parameter $\epsilon$, ermöglicht den Nutzern davon eins auszusuchen. Der Nutzer kann somit sein Parameter gerichtet an seiner Anfrage auswählen.


Dieses Paper zeigt mögliche Schwierigkeiten bei der Evaluation dieser Bachelorarbeit. Es wird eine einfache Anfrage (Durchschnitt) durchgeführt und der Laplace Mechanismus verwendet. Es wird auch keine Menge an $\epsilon$- Budget genutzt. Es wird eine große Menge an Anfragen stattfinden (1000 mal). Die Anwendbarkeit des Verfahrens auf den medizinischen Fall kann entsprechend der Ergebnisse beschränkt genutzt werden.

Insgesamt fehlt in der medizinischen Forschung die Untersuchung von Frameworks für das Gesundheitswesen. Es gibt vereinzelt Versuche, vor allem für Machine Learning, jedoch fehlt es an einer Evaluation in den Aspekten von Genauigkeit und Privatisierung von \gls{dp} Frameworks für die Medizin.


