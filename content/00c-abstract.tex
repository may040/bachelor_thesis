\addchap*{Kurzfassung}
\markboth{Kurzfassung}{}

Die aktuelle Bundesregierung aus SPD, FDP und Grünen führt im Zuge der Digitalisierung der Medizin nach dem Opt-Out Prinzip die \gls{ePa} ein. Nun erhält jeder Krankenkassenversicherte eine \gls{ePa}, die die vollständige Dokumentation der medizinischen Behandlungshistorie umfasst. Die Analyse solcher großen Datenbestände hat das Potenzial, die medizinische Forschung ganz erheblich voranzubringen und könnte neue Grundlagen zur Entwicklung von Behandlungsmöglichkeiten schaffen. Die Nutzung solcher sensiblen Daten ist unter Einhaltung einer Anonymisierung möglich. Trotz des Entfernens identifizierender Merkmalen kann es zu einer Re-Identifizierung durch Hintergrundwissen kommen. Der Schutz der Privatsphäre steht bei solchen Daten im Vordergrund.  
Ein zunehmend beachteter Ansatz zum Schutz der Privatsphäre bei der Auswertung von Daten ist \gls{dp}. Es fügt den Originaldaten mathematisches Rauschen hinzu. Solch eine Privatisierung der Daten soll die Gefahr einer Re-Identifizierung eines einzelnen Eintrages im Datensatz minimieren. Somit wird die Auswertung der sensiblen Daten für die Forschung unter dem Schutz der Privatsphäre möglich. In der Praxis hat dieses Verfahren \gls{dp} schon an Bedeutung gewonnen und wird in verschieden Teilgebieten der Medizin und anderen eingesetzt.

Das Ziel in der vorliegenden Arbeit ist es, die drei \gls{dp} Frameworks Smartnoise SDK (Microsoft), Google \gls{dp} und IBM \gls{dp} für den Einsatz in einem medizinischen Anwendungsfall zu evaluieren. Zunächst wird das grundlegende Wissen über \gls{dp} sowie den Metriken zur Evaluation vermittelt. Diese umfassen die Kategorien Privatsphäre, Genauigkeit und Erwartungstreue. Weiterhin werden die Erkenntnisse verwandter Arbeiten bezüglich \gls{dp} in der Genauigkeit sowie Anwendung für die Medizin dargestellt. Des Weiteren werden die drei Frameworks in ihrer Gesamtkonstruktion beschrieben. Anschließend wird konkret der Einsatz der Frameworks für die medizinische Forschung skizziert. 
Im Rahmen dieser Arbeit wird eine generische Schnittstelle für den Anwendungsfall implementiert, in der die verrauschten Durchschnittswerte durch die Metriken evaluiert werden. Die resultierenden Ergebnisse sollen Aufschluss über die Einsatzfähigkeit der Frameworks für den Use-Case mit Gesundheitsdaten geben.