\chapter{Ausblick und Fazit}
In diesem Kapitel folgt das Fazit der gesamten Bachelorarbeit, wobei es eine Zusammenfassung der Kapitel, Evaluationsresultate und die Beantwortung der Aufgabenstellung dieser Arbeit umfasst. Die zeitliche und inhaltliche Tragweite einer Bachelorarbeit ist formell beschränkt, sodass ein Ausblick für weitere Forschung folgt. 
\section{Fazit}
Die Digitalisierung des Gesundheitswesens wird durch die \gls{ePa} gefördert. In dieser werden die medizinischen Daten eines Patienten hinterlegt. Bisher fehlen jedoch technische Lösungen für den Schutz der Privatsphäre der betroffenen Patienten. Die \gls{ePa} stellt für die medizinische Forschung ein großes Potenzial an Nutzbarkeit dar. In dieser Arbeit wurden dafür drei Frameworks von bekannten Softwarehersteller für die datenschutzgerechte Verarbeitung von sensiblen Gesundheitsdaten evaluiert. Ihre Ausgabe erfolgt nach dem Prinzip von \gls{dp} mit dem dafür eingesetzten Laplace Mechanismus. 

In einem an die medizinische Forschung angelegten Anwendungsfall evaluiert diese Arbeit die Frameworks IBM \gls{dp}, Smartnoise SDK und Google \gls{dp} auf ihre Einsetzbarkeit. Er umfasst ihre Verbindung zu einem zentralen Server mit Gesundheitsdaten. Auf sie haben die Forscher durch die Frameworks einen Zugriff, um durch die Analyse Erkenntnisse zu erlangen und Statistiken zu veröffentlichen unter dem Schutz der Privatsphäre der Patienten.

Für diese spezifische Anwendung wurde eine prototypische generische Schnittstelle implementiert. Sie liest die erzeugten Daten, die durch die Wahrscheinlichkeitsverteilung der COVID-19 infizierten Menschen aus dem Monat Dezember des Jahres 2021 randomisiert generiert worden sind, ein. Jedes Framework führt ihre \gls{dp} Funktion \textit{Durchschnitt} auf diesen Daten durch. Für die Evaluation dieser Ausgabe sind verschiedene Metriken in den Kategorien Privatsphäre, Genauigkeit und Erwartungstreue aus Literaturquellen herangezogen worden. Sie sind auf ihre Aussagekraft beurteilt und dementsprechend verwendet worden. Ein weiterer Aspekt dieser Evaluation ist die Ausführungszeit der Berechnungen gewesen.

Bei den resultierenden Werten der Metriken werden teilweise unerklärliche Unterschiede festgestellt. Diese sind auf die Implementierung der Frameworks zurückzuführen. Folgende Schlüsse lassen sich aus den Ergebnissen ziehen:
Das Framework IBM \gls{dp} entspricht in allen Kategorien den Erwartungen und weist keine unerwarteten Ergebnisse auf. Google \gls{dp} hingegen fällt in allen am schlechtesten aus und sorgt für eine hohe Ungenauigkeit in den Daten. Smartnoise SDK hat in allen Bereichen mittlere Werte mit einer gewissen Instabilität. In der Performance erfolgen die Ausführungszeiten bei IBM \gls{dp} und Google \gls{dp} im Sekundenbereich, in Kontrast dazu bei Smartnoise SDK im Stundenbereich.

Nach diesen Ergebnissen der Evaluation im medizinischen Anwendungsfall hebt sich das Framework IBM \gls{dp} mit seiner Leistung deutlich ab. Es erzielt einen hohen Grad an Privatsphäre und Nutzbarkeit der Daten in einer akzeptablen Anwendungszeit. Somit lässt es sich für diesen Anwendungsfall einsetzen. Auch wenn Smartnoise SDK in einem instabilen mittleren Niveau Daten privatisiert, scheitert es bei der Anwendung an seiner Ausführungszeit. Für Forscher ist sie zu groß, um dauerhaft Mehrfachausführungen und Auswertungen durchzuführen zu können. Ebenfalls scheitert das Google \gls{dp} an seiner geringen Qualität an Schutz für die Privatsphäre der Daten. Zusätzlich existiert die hohe Ungenauigkeit der Daten, wodurch ihre Nutzbarkeit verfällt.
\section{Ausblick}
In der Evaluation sind die Resultate aufgrund fehlender Richtwerte in diesen Metriken relativ in ihren Kategorien zugeordnet worden. Somit erfolgte keine absolute Einordnung, da Vergleichsdaten fehlen. Es fehlt in der medizinischen Forschung an Experimenten bzw. Versuchen mit Frameworks, um die Auswertung strukturierter und aussagekräftiger zu gestalten. Hierfür sind Richtwerte der Metriken erforderlich. Dies wurde in dieser Arbeit bei den Folgerungen (Kapitel 7) erwähnt. Die Schwierigkeit liegt darin, einen Schwellwert für die Akzeptanz und Nicht-Akzeptanz des Wertes zu definieren. 

Um ein Verständnis für die Metriken zu entwickeln, helfen die Auswertungen verschiedener \gls{dp} Funktionen. In dieser Evaluation wurde der Durchschnitt von Alterswerten betrachtet. Die Frameworks sind mit statistischen Funktionen wie z.B. Summe, Standardabweichung usw. sowie komplexen Modellen wie beim maschinellen Lernen ausgestattet. Ein Vergleich in diesen könnte weitere Erkenntnisse über die metrischen Werte liefern.

In Kapitel 2 werden verschiedene Mechanismen fürs Verrauschen genannt und es gibt weitere die in den Frameworks vorhanden sind. Ein allseits bekannter ist der Laplace Mechanismus und ein weiterer ist der Gauß Mechanismus. Das mathematische Rauschen kann unterschiedlich ausfallen, da sie verschiedene Verteilungen folgen. So benötigt der Gauß Mechanismus zum Beispiel zusätzlich den Parameter $\delta$, welcher den Grad an Verletzung der Anforderungen bestimmt. In einer weiteren Forschung können diese Mechanismen in einem Vergleich evaluiert werden. Eine umfangreiche Untersuchung ihrer Stärken kann Erkenntnisse über ihre Anwendung für spezifische Themenbereiche folgen lassen.

Schlussendlich könnte diese Bachelorarbeit als Grundlage einer prototypischen Umsetzung des Frameworks IBM \gls{dp} in einer praktischen Forschung dienen. In diesem Fall könnten bereits implementierten Metriken dieser Arbeit in Python zur Evaluierung herangezogen werden. Hierfür genügen lediglich die Ergebnisse des Frameworks in einer CSV Datei abzuspeichern, welche nach dem Einlesen durch die generische Schnittstelle ausgewertet werden kann. Bei der praktischen Anwendung ist speziell die Überprüfung der Qualität des Frameworks in der Privatisierung und Performance zu überprüfen. Eine praktische Nutzung des Frameworks kann zu einer erkenntnisreichen Evaluation führen.